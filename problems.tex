\documentclass{article}
\usepackage[utf8]{inputenc}
\usepackage[english]{babel}
\usepackage{hyperref}
\usepackage{indentfirst}
\usepackage{amsmath}
\usepackage{chemfig}

\title{Chemistry Problems}

\setlength{\parindent}{2em}
\setlength{\parskip}{1em}

\begin{document}

\maketitle

\section{\href{https://www.khanacademy.org/science/ap-chemistry}{Khan Academy: AP Chemistry}}

\begin{enumerate}
    \item Calculate the number of moles in a 1.52kg sample of glucose (\chemfig{C_6H_{12}O_6)}
    
    \textbf{Solution:} Atomic mass of carbon is 12.011u, hydrogen 1.008u, and oxygen 15.999u.
    
    \(12.011\cdot6+1.008\cdot12+15.999\cdot6=180.156\)u molecular mass of glucose.
    
    \[1.52\text{kg glucose} \cdot \frac{1000\text{g}}{1\text{kg}} \cdot \frac{1\text{mol glucose}}{180.156\text{g}}=8.44\text{ moles of glucose}\]
    
    \item A certain ion has 20 protons and 18 electrons. What kind of element is this ion, and what is its net charge?
    
    \textbf{Solution:} It's a calcium ion with a net charge of +2.
    
\end{enumerate}


\end{document}
